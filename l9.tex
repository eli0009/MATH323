\documentclass{article}
\usepackage{amssymb}
\usepackage{amsfonts}
\usepackage{amsmath}
\usepackage{cancel}
\author{Enlai Li}
\title{MATH323 -- Lecture 9}
\date{February 2, 2023}
\begin{document}
\maketitle
\section{Geometric probability mass function (p.m.f.)}
\begin{align*}
    P(Y=k) = {(1-p)}^{k-1}p          \text{ where } k = 1, 2, \dots                         \\
    \sum_{k=1}^{\infty} P(Y=k) = 1                                                          \\
    \mathbb{E}(Y) & = \sum_{k=1}^{\infty} kP(Y=k)                                           \\
                  & = \sum_{k=1}^{\infty} (1-p)^{k-1}p                                      \\
                  & = p \sum_{k=1}^{\infty} kq^{k-1}                                        \\
                  & = p \sum_{k=1}^{\infty} \frac{d}{dq}(q^k)                               \\
    \sum_{n=i}^{\infty}a_nx ^{n} \text{\indent note: } \sum_{n=1}^{\infty} \frac{(-1)^n}{n} \\
                  & = p \frac{d}{dq}[\sum_{k=1}^{\infty} q^k]                               \\
                  & = p \frac{d}{dq}[\frac{q}{1-q}]                                         \\
                  & = p \frac{1 \times (1-q) + q}{(1-q)^2}                                  \\
                  & = p \frac{1 }{(1-q)^2}                                                  \\
                  & = \cancel{p} \frac{1}{p^x}                                              \\
                  & = \frac{1}{p}                                                           \\
    \mathbb{E}(Y) = \frac{1}{p}
\end{align*}

\begin{align*}
    \binom{49}{6}                                      \\
    p = \frac{1}{\binom{49}{6}}                        \\
    \mathbb{E}(Y) = \binom{49}{6}                      \\
    2 \times 52.5 = 105                                \\
    \binom{49}{6} = 13,983,816                         \\
    \frac{\binom{49}{6}}{105} = \frac{13,983,816}{105} \\
    = 133,180 \text{ games}
\end{align*}

\begin{align*}
    V(Y)               & = \mathbb{E}(Y^2) - [\mathbb{E}(Y)]^2         \\
    \mathbb{E}(Y^2)    & = \mathbb{E}[Y(Y-1)] + \mathbb{E}[Y]          \\
    \mathbb{E}[Y(Y-1)] & = \sum_{k=1}^{\infty}k(k-1)P(y=k)             \\
                       & = \sum_{k=1}^{\infty}k(k-1)q ^{k-1}p          \\
                       & = pq \sum_{k=1}^{\infty}k(k-1)q ^{k-2}        \\
                       & = pq \frac{d^2}{dq^2}[\sum_{k=1}^{\infty}q^k] \\
    V(Y) = \frac{(1-p)}{p^2}
\end{align*}

\subsection{Negative Binomial Distribution}
Notation: $Bin(n,p)$, $Geo(P)$, $NB(r,p)$
\[
    X_i = \begin{cases}
        1 & p   \\
        0 & 1-p
    \end{cases}
\]
\underline{r} successes:
\begin{gather*}
    r \\
    Y = K
\end{gather*}
There is at least 2 success. Consider K boxes filled with 0, there is a success box
with 1 at position K. There is $r-1$ boxes, corresponding to $k-1$ boxes, it then becomes
\begin{gather*}
    \binom{k-1}{r-1}p^rq ^{k-r}                                                        \\
    P(Y=k)             = \binom{k-1}{r-1}p^rq ^{k-r} \text{ where } k = r, r+1, \dots \\
    q                  = 1-p                                                          \\
    \boxed{(\frac{1}{1-x})^r  = \sum_{k=r}^{\infty} \binom{k-1}{r-1} x ^{k-r} \text{ where }
        0 < x < 1} \\
    \sum_{k=r}^{\infty} P ( Y=K) = p^r \sum_{k=r}^{\infty}\binom{k-1}{r-1} q ^{k-r} \\
    = p^r (\frac{1}{1-q})^r \\
    =1
\end{gather*}

If $Y \sim NB(r,p)$ then
\begin{gather*}
    \mathbb{E}(Y) = \frac{r}{p} \\
    V(Y) = \frac{r(1-p)}{p^2} \\
    Y \sim NB (r, p) \\
    Y = \sum_{i=1}^{r} X_i \text{ where } X_i \stackrel{ind}{\sim} Geo(P)
\end{gather*}

If $Y \sim Bin(n,p)$ then $Y = \sum_{i=1}^{n}X_i$,
$
    X_i = \begin{cases}
        1 & p   \\
        0 & 1-p
    \end{cases}
$

\subsection{HyperGeometric Distribution}
$R + B = N$:
\[
    \boxed{R \text{\indent } B}
\]
Y = \# of rectangles in a sample of size n
\begin{gather*}
    S \text{\indent } \binom{N}{n} \\
    P(Y=r) \\
    A = \binom{R}{r}\binom{B}{n-r} \\
    P(Y=r) = \frac{\binom{R}{r}\binom{B}{n-r}}{\binom{N}{n}} \text{, where } r = 0,1,2,\dots , n \\
    Y \sim H \mathbb{G}(N,M,n) \\
    P(Y=r) = \frac{\binom{M}{r}\binom{N-M}{n-r}}{\binom{N}{n}} \text{, where }
    r = 0,1,2, \dots , n
\end{gather*}

Take $n$ balls:
\begin{gather*}
    Y = \sum_{i=1}^{n} X_i \\
    X_i = \begin{cases}
        1 \\
        0
    \end{cases}
\end{gather*}

\end{document}